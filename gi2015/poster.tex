\documentclass[final]{beamer}
\mode<presentation> {  %% check http://www-i6.informatik.rwth-aachen.de/~dreuw/latexbeamerposter.php for examples
  \usetheme{I6pd}    %% you should define your own theme e.g. for big headlines using your own logos 
}
\usepackage[english]{babel}
\usepackage[latin1]{inputenc}
\usepackage{amsmath,amsthm, amssymb, latexsym}
\usepackage{array,booktabs,tabularx}
\usepackage{graphicx}
\usefonttheme[onlymath]{serif}
\boldmath
\usepackage[orientation=landscape,size=custom,width=121,height=121,scale=1.5,debug]{beamerposter}
\title{\huge Building More Expressive (and Portable) Galaxy Workflows}
\author[Chilton, Galaxy Project]{John Chilton.\textsuperscript{*1} and The Galaxy Team}
\institute[]{\textsuperscript{1}Department of Biochemistry and Molecular Biology, The Pennsylvania State University, PA, USA}
\date{October 29th, 2015}
  
\usepackage{poster2015}
\usepackage{workflows}

\begin{document}
\begin{frame}
  \begin{columns}
    \begin{column}{.50\textwidth}
      \begin{beamercolorbox}[center,wd=\textwidth]{postercolumn}
        \begin{minipage}[T]{.97\textwidth}  % tweaks the width, makes a new \textwidth
          \parbox[t][\columnheight]{\textwidth}{
            \galaxyintroblock
            \vfill
            \workflowsblock
            \vfill
            \collectionsblock
            \vfill
            \rnaseqblock
            \vfill
            \corephylogenomics
          }
        \end{minipage}
      \end{beamercolorbox}
      \end{column}

    \begin{column}{.50\textwidth}
      \begin{beamercolorbox}[center,wd=\textwidth]{postercolumn}
        \begin{minipage}[T]{.98\textwidth} % tweaks the width, makes a new \textwidth
          \parbox[t][\columnheight]{\textwidth}{
            \openms
            \vfill
            \workflowrewrite
            \vfill
            \workflowrewriteapps
            \vfill
            \futurework
            \vfill
            \cwl
            \vfill
            \cwlingalaxy
          }
        \end{minipage}
      \end{beamercolorbox}
    \end{column}              

  \end{columns}   
\end{frame}
\end{document}
