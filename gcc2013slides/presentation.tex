%\documentclass[notes,handout]{beamer}
\documentclass[]{beamer}
\usepackage{amsmath,amsthm,ifthen}
\usepackage{uofmtheme}
\usepackage{pgfpages}
\usepackage{graphicx}
\usepackage{galaxyextensionscontent}
\usepackage{cloudcomputingcontent}
\usepackage{array,booktabs,tabularx}

\newcommand {\flexblock}[2] {
    \frame{
    \frametitle{#1}
    #2
    }
}

\newcommand {\fleximage}[2] {
\includegraphics[keepaspectratio,width=#1\textwidth]{../images/#2}
}

\newcommand {\screen}[1]{
\includegraphics[keepaspectratio,width=\textwidth]{../images/#1}   
}

\newcommand {\screenh}[1]{
    \begin{center}
        \includegraphics[keepaspectratio,height=.9\textheight]{../images/#1}   
    \end{center}
}

\title{Galaxy-P: Beyond Proteomics}
\author{John Chilton}


\begin{document}
\frame{\titlepage}
\note[] {
}

\frame{
    \frametitle{Three Ways to Galaxy-P}
    usegalaxyp.org
    getgalaxyp.org
    biocloudcentral.msi.umn.edu
}

\frame{
    \frametitle{Tools}
    Helped Ira Cooke fill out the proteomics tool shed. Proteomics is now third most populate category.
}

\frame{
    \frametitle{Workflows}
    We have built noodly appendages for sophisticate protein identification and quantitation techniques, as well as the emerging application areas of proteogenomics and metaproteomics.
}

\frame{
    \frametitle{Beyond Proteomics?}

}

\frame{
    \frametitle{LWR}
    Run Galaxy jobs on Windows machines.
    \begin{itemize}
        \item Galaxy-style Stack (Paste-based Python application).
        \item Secure.
        \item Available in Galaxy today, just another job destination.
        \item Well documented (https://lwr.readthedocs.org/).
        \item Cross-platform.
    \end{itemize}
}

\frametitle{
    \frametitle{LWR - }
}


\end{document}