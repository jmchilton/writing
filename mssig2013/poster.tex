\documentclass[final]{beamer}
\mode<presentation> {  %% check http://www-i6.informatik.rwth-aachen.de/~dreuw/latexbeamerposter.php for examples
  \usetheme{Berlin}    %% you should define your own theme e.g. for big headlines using your own logos 
}
\usepackage[english]{babel}
\usepackage[latin1]{inputenc}
\usepackage{amsmath,amsthm, amssymb, latexsym}
\usepackage{array,booktabs,tabularx}
\usefonttheme[onlymath]{serif}
\boldmath
\usepackage[orientation=landscape,size=custom,width=100,height=100,scale=2,debug]{beamerposter}
\title[Innovative, Reproducible MS-Based Proteomic Informatics in the Cloud for Emerging Applications with Galaxy-P and CloudBioLinux]{}
\author[John M. Chilton; James E. Johnson; Ebbing P. de Jong; Getiria Onsongo; Benjamin J.
Lynch; Pratik D. Jagtap; Timothy J Griffin]{}
\institute[]{University of Minnesota Supcomputing Institute}
\date{June 10th, 2013}

\newlength{\columnheight}
\setlength{\columnheight}{105cm}

\begin{document}
\begin{frame}
  \begin{columns}

    \begin{column}{.49\textwidth}
      \begin{beamercolorbox}[center,wd=\textwidth]{postercolumn}
        \begin{minipage}[T]{.95\textwidth}  % tweaks the width, makes a new \textwidth
          \parbox[t][\columnheight]{\textwidth}{
            \begin{block}{Galaxy-P Overview}
              \begin{itemize}
              \item Built on Galaxy framework - popular and widely cited in the genomics community.
                \begin{itemize}
                  \item Open source, web-based platform.
                  \item Build complex, reproduible 
                \end{itemize}
              \item Integrated dozens of tools for proteomic analysis.
              \item Built complex workflows for protein identification, quantification, and proteogenomic analysis.
              \end{itemize}
            \end{block}
            \vfill
            % TODO: 
            \begin{block}{Proteogenomics Workflow for Galaxy-P}
              Galaxy-P provides an integrated platform for every step of proteogenomic analysis.
              \begin{itemize}
                \item Build target database - download and translate EST databases or perform gene prediction with Augustus.
                \item Numerous tools for identification. 
                \item Workflow utilizing BLAST to identify novel peptides.
                \item Tool to assess peptide-spectrum matches and visualize spectra.
                \item Visualize identified, novel peptides on the genome.
              \end{itemize}
            \end{block}
            \vfill
            \begin{block}{Galaxy-P in the Cloud}
              Spin up your own cluster for Galaxy-P in the cloud.
            \end{block}
            \vfill
          }
        \end{minipage}
      \end{beamercolorbox}
    \end{column}

    \begin{column}{.49\textwidth}
      \begin{beamercolorbox}[center,wd=\textwidth]{postercolumn}
        \begin{minipage}[T]{.95\textwidth} % tweaks the width, makes a new \textwidth
          \parbox[t][\columnheight]{\textwidth}{
            \begin{block}{CloudBioLinux}
              \begin{itemize}
                \item Platforms for systems biology like Galaxy-P must bring together many diverse bioinformatic tools. Installing, configuring, and maintaining these tools is a real challenge.
                \item CloudBioLinux provides a 
            \end{block}
            \vfill
            \begin{block}{CloudBioLinux - Desktop Tools}
              \item OpenMS and TOPPAS
              \item PeptideShaker and SearchGUI
              \item MZMine
            \end{block}
            \vfill
           
          }
        \end{minipage}
      \end{beamercolorbox}
    \end{column}              

  \end{columns}   
\end{frame}
\end{document}
