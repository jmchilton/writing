%\documentclass[notes,handout]{beamer}
\documentclass[]{beamer}
\usepackage{amsmath,amsthm,ifthen}
\usepackage{uofmtheme}
\usepackage{pgfpages}
\usepackage{graphicx}
\usepackage{galaxyextensionscontent}
\usepackage{cloudcomputingcontent}
\usepackage{array,booktabs,tabularx}

\newcommand {\flexblock}[2] {
    \frame{
    \frametitle{#1}
    #2
    }
}

\newcommand {\fleximage}[2] {
\includegraphics[keepaspectratio,width=#1\textwidth]{../images/#2}
}

\newcommand {\screen}[1]{
\includegraphics[keepaspectratio,width=\textwidth]{../images/#1}   
}

\newcommand {\screenh}[1]{
    \begin{center}
        \includegraphics[keepaspectratio,height=.9\textheight]{../images/#1}   
    \end{center}
}

\title{Galaxy-P Update: Pre ASMS 2013}
\author{John Chilton}


\begin{document}
\frame{\titlepage}
\note[] {
}

\frame{
    \frametitle{A Recycled Presentation}
    \begin{itemize}
        \item ASMS Poster - "Reproducible Proteomic Workflows using Extensions to the Galaxy Framework"
        \item Galaxy-P User Guide\\
            \begin{itemize}
                \item http://bit.ly/usegalaxyp-guide - Version for public server.
                \item http://bit.ly/usegalaxyp-guide-pdf
                \item http://bit.ly/msi-galaxyp-guide - Version for MSI server. 
                \item http://bit.ly/msi-galaxyp-guide-pdf
                \item Intermidate topics that take over where Pratik's Galaxy-P 101 ends.
            \end{itemize}
        \item ASMS Poster - "Building Proteomic Application Platforms for Cloud Computing Environments with CloudBioLinux"
    \end{itemize}
}

\frame{
    \frametitle{Reproducible Proteomic Workflows using Extensions to the Galaxy Framework}
    \screen{asms2013poasterge.png}
}

\introblock
\galaxypblock
\galaxypcloudblock

\section{Galaxy-P: Dealing with Many Files}
\manyfileschallengesblock
\frame{
    \frametitle{JGalaxy} 
    \jgalaxydesc
}

\frame{
    \frametitle{JGalaxy Launch}
    \jgalaxylaunchfig[.8]
}

\frame{
    \frametitle{JGalaxy Download} 
    \jgalaxydownloadfig[.8]
}

% \toolextensionsblock

\frame{
    \frametitle{Selecting Multiple Files (1 / 3)}
    \toolmgfrepeatfig[.75]
}

\frame{
    \frametitle{Selecting Multiple Files (2 / 3)}
    \toolmultipledesc
}

\frame{
    \frametitle{Selecting Multiple Files (3 / 3)}
    \toolmultipleselectfig[.75]
}

\frame{
    \workflowlimitfigs
}

\frame{
    \multiplefiledatasetsdesc
}

\frame{
    \openmsfig    
}

\frame{
    \frametitle{Detouring to Galaxy-P User Guide Walkthrough}
    Detour away from high-level poster content, into screenshots from the Galaxy-P User Guide, to answer
    two big questions!
    \begin{itemize}
        \item How can one create multiple file datasets?
        \begin{itemize}
            \item FTP, TINT Export, URL Downloads, Web Browser Uploads. 
        \end{itemize}
        \item How can one use multiple file datasets?
        \begin{itemize}
            \item Protein identification of multiple file datasets.
        \end{itemize}
    \end{itemize}
}

\frame{
    \frametitle{Uploading Files via FTP}
    \screenh{winscp_msi_galaxyp_login.png}
}

\frame{
    \screen{winscp_save_host.png}
}

\frame{
    \screen{winscp_password.png}
}

\frame{
    \screen{winscp_about_to_copy.png}
}

\frame{
    \screen{winscp_copying.png}
}

\frame{
    \screen{winscp_copied.png}
}

\frame{
     \frametitle{Multiple File FTP Upload Tool}
     \screenh{ftp_multifile_upload_tool.png}
}

\frame{
     \frametitle{A New Multiple File Dataset}
     \screen{multiple_file_raw.png}
}

\frame{
    \frametitle{Creating Multiple File Datasets via TINT Export}
    \screen{multifile_tint_export.png}
}

\frame{
    \frametitle{Creating Multiple File Datasets via Browser Upload}
    \screen{jgalaxy_multifile_upload.png}
}


\frame{
    \frametitle{Creating Multiple File Datasets via URL Downloads}
    \screen{multifile_url_paste.png}
}

\frame{
    \frametitle{Creating Multiple File Datasets via Browser Upload}
    \screenh{ftp_multifile_upload_tool.png}
}

\frame{
    \frametitle{Using Multiple File Datasets}
    Walk through an example using OMSSA and Scaffold and demonstrate the benefits.
    \screen{scaffold_start.png}
}

\frame{
    \frametitle{Example Dataset for Scaffold}
    \begin{itemize}
        \item Three fractions of the "Mycobacterium tuberculosis culture filtrate LC-MS/MS" dataset on proteomexchange.
        \item {\tiny http://proteomecentral.proteomexchange.org/cgi/GetDataset?ID=PXD000111}
        \item Available to everyone in the Galaxy-P data library called "Example Data"
        \item User guide covers how to import this dataset into a new history.
    \end{itemize}
}

\frame{
    \screenh{scaffold_dataset_imported.png}
}

\frame{
    \screenh{scaffold_workflow_published.png}
}

\frame{
    \screenh{scaffold_workflow_view_import.png}
}

\frame{
    \screenh{scaffold_workflow_use.png}
}

\frame{
    \screenh{scaffold_workflow_imported.png}
}

\frame{
    \screenh{scaffold_workflow_run.png}
}

\frame{
    \screenh{scaffold_raw_convert.png}
}

\frame{
    \screenh{scaffold_run_omssa.png}
}

\frame{
    \screenh{scaffold_run.png}
}

\frame{
    \screenh{scaffold_download_sf3.png}
}

\frame{
    \screen{scaffold_samples.png}
}

\frame{
    \frametitle{Why not stop at Scaffold and ProteinPilot?}
    \screen{identification_pipelines}
}

\frame{
    \frametitle{PeptideShaker (and SearchGUI)}
    \begin{itemize}
        \item Automatically search X! Tandem and OMSSA.
        \item Combine results and perform FDR analysis at protein, peptide, and/or PSM level.
        \item A-score
    \end{itemize}
}

\frame{
    \frametitle{Example Dataset for PeptideShaker}
    \begin{itemize}
        \item Three fractions of the "Mycobacterium tuberculosis ATP-binding Proteome" dataset on proteomexchange.
        \item {\tiny http://proteomecentral.proteomexchange.org/cgi/GetDataset?ID=PXD000141}
        \item Available publically on usegalaxyp.org in a data library called "Example Data"
        \item User guide covers how to import this dataset into a new history.
    \end{itemize}
}

\frame{
    \screenh{peptideshaker_dataset_imported.png}
}

\frame{
    \screenh{peptideshaker_database_download.png}
}

\frame{
    \screenh{peptideshaker_create_target_decoy.png}
}

\frame{
    \screenh{peptideshaker_convert_mgf.png}
}

\frame{
    \screenh{peptideshaker_run.png}
}

\frame{
    \screenh{peptideshaker_table.png}
}


\section{Galaxy-P: Cross-Platform Job Execution}
\windowschallangesblock
\lwrblock

\frame{
    \frametitle{Cross Platform Tool - msconvert (1 / 2)}
    \msconvertfig[.8]
}

\frame{
    \frametitle{Cross Platform Tool - msconvert (2 / 2)}
    \msconvertdesc
}

% \maxquantblock
\frame {
    \frametitle{Cross Platform Tool - MaxQuant (1 / 3)}
    \maxquantfig[.5]
}

\frame {
    \frametitle{Cross Platform Tool - MaxQuant (2 / 3)}
    \maxquantdesc
}

\frame {
    \frametitle{Cross Platform Tool - MaxQuant (3 / 3)}
    \maxquantworkflow[.5]
}

% \proteinpilotblock
\frame{
    \frametitle{Cross Platform Tool - ProteinPilot (1 / 3)}
    \proteinpilotfig[.5]
}

\frame{
    \frametitle{Cross Platform Tool - ProteinPilot (2 / 3)}
    \proteinpilotdesc
}

\frame{
    \frametitle{Cross Platform Tool - ProteinPilot (3 / 3)}
    \proteinpilotworkflow[.5]
}

\section{CloudBioLinux for Data Analysis}
\frame{
    \frametitle{Building Proteomic Application Platforms for Cloud Computing Environments with CloudBioLinux}
    \screen{asms2013poastercc.png}
}

\cloudintroblock
\biocloudcentralblock
\cblappsblock
\cblplsblock

\frame{
    \frametitle{Graphical Applications}
    \cblguiappdesc
}

\section{CloudBioLinux for Platform Development}
\cblplatformblock

\frame{
    \frametitle{CloudBioLinux as a Platform - wine (1 / 2)}
    \cblwinedesc
}

\frame{
    \frametitle{CloudBioLinux as a Platform - wine (2 / 2)}
    \cblwinefig[.5]
}

\frame{
    \frametitle{CloudBioLinux as a Platform - Galaxy-P}
    \cblgalaxypdesc
}

\end{document}